\section{R and pbdR installation on Cray system }\label{sec:Rinstall}

Below is installation instructions of \proglang{R} on Cray machine. 
Prerequisites for R and pbdR software installation on Cray machine: GCC, Cray-MPICH and ACML/MKL.

We use following software tools/packages: Programming Environment GNU-4.2.34 with \pkg{GCC}-4.8.2, \pkg{Cray-MPICH}-6.2.0, \pkg{ACML}-5.3.1, and \proglang{R} 3.0.2(compied dynamically). The \proglang{R} packages and their corresponding versions used in these: \pkg{rlecuyer} 0.3-3, \pkg{memuse} 0.2-0, \pkg{pbdMPI} 0.2-2, \pkg{pbdSLAP} 0.1-8, \pkg{SEXPtools} 0.1-0, \pkg{pbdBASE} 0.3-0, and \pkg{pbdDMAT} 0.3-0.

\vspace*{-.5cm}
\begin{lstlisting}[language=bashy]

## Load/set appropriate modules and software environments 
module swap PrgEnv-cray PrgEnv-gnu/4.2.34
module load java/jdk1.7.0_45
module load acml/5.3.1

## Create installation directory on lustre space
mkdir -p /scratch/sciteam/\$USER/R-3.0.2-Install
export R_WORK_HOME=/scratch/sciteam/\$USER/R-3.0.2-Install
cd \$R_WORK_HOME

## Download R source code 
wget http://mirrors.nics.utk.edu/cran/src/base/R-3/R-3.0.2.tar.gz
tar -xzvf R-3.0.2.tar.gz
cd R-3.0.2

## Configure R
./configure --prefix=\$R_WORK_HOME --enable-R-profiling --enable-memory-profiling --enable-R-shlib --enable-BLAS-shlib --enable-lto --enable-byte-compiled-packages --enable-shared --enable-long-double --with-readline --with-tcltk --with-cairo --with-libpng --with-jpeglib --with-libtiff --with-system-pcre --with-valgrind-instrumentation --with-x --with-blas=``-I/opt/acml/5.3.1/gfortran64_mp/include -L/opt/acml/5.3.1/gfortran64_mp/lib -lacml_mp'' --with-lapack > configure.log

make > make.log
make check > make_check.log
make install > make_install.log
make check-all > make_check_all.log

## Install R packages
export PATH=\$R_WORK_HOME/bin:\$PATH
export LD_LIBRARY_PATH=\$R_WORK_HOME/lib64/R/lib:\$LD_LIBRARY_PATH

mkdir -p \$R_WORK_HOME/R_pkg_sources
cd \$R_WORK_HOME/R_pkg_sources


## Download R packages
wget http://cran.r-project.org/src/contrib/rlecuyer_0.3-3.tar.gz
wget http://cran.r-project.org/src/contrib/pbdMPI_0.2-2.tar.gz
wget http://cran.r-project.org/src/contrib/pbdSLAP_0.1-8.tar.gz
wget https://github.com/wrathematics/pbdBASE/archive/master.zip
mv master pbdBASE.zip
unzip pbdBASE.zip

wget https://github.com/wrathematics/pbdDMAT/archive/master.zip
mv master pbdDMAT.zip
unzip pbdDMAT.zip

wget  https://github.com/wrathematics/SEXPtools/archive/master.zip
mv master SEXPtools.zip
unzip SEXPtools.zip

wget https://github.com/wrathematics/memuse/archive/master.zip
mv master memuse.zip
unzip memuse.zip

## R packages installation
R CMD INSTALL --no-test-load rlecuyer_0.3-3.tar.gz

## pbdMPI install
R CMD INSTALL --no-test-load pbdMPI --configure-args=``--with-mpi=/opt/cray/mpt/6.2.0/gni/mpich2-gnu/48/ --with-mpi-type=MPICH3''

R CMD INSTALL --no-test-load pbdSLAP_0.1-8.tar.gz
 
R CMD INSTALL --no-test-load pbdBASE

R CMD INSTALL --no-test-load pbdDMAT

R CMD INSTALL --no-test-load SEXPtools

R CMD INSTALL --no-test-load memuse

## Copy all needed dynamic libraries to Lustre 

mkdir -p \$R_WORK_HOME/system_libs
cd \$R_WORK_HOME/
./dyn_libs_copy.sh \$R_WORK_HOME/lib64/R/lib \$R_WORK_HOME/system_libs
./dyn_libs_copy.sh \$R_WORK_HOME/lib64/R/library/pbdMPI/libs \$R_WORK_HOME/system_libs

echo ``Now you are ready to use \proglang{R} and \proglang{pbdR}. Good Luck !!!''

#*********************
#*********************
## Below is source code of `'dyn_libs_copy.sh''
## This script uses ldd and copy all needed shared object files recursively
OPTIONS_HELP='
Command line options:
     -h Print this help menu
     -v Verbose (lookup_dir_Path destination_dir_path)

Usage: ./dyn_libs_copy.sh lookup_dir_Path destination_dir_path
Usage Example:
./dyn_libs_copy.sh /package/version/os_compiler/bin /package/version/os_compiler/system_libs
./dyn_libs_copy.sh -v /package/version/os_compiler/lib /package/version/os_compiler/system_libs

Verbose mode: ./dyn_libs_copy.sh -v lookup_dir_Path destination_dir_path
'

# Debug function
log(){
if [[ \$debug -eq 1 ]]; then
        echo ``\$@''
    fi
}

## recur_copy function: takes two arguments: filename and Destination path
recur_copy()
{
    log -e ``#####\nFile Name(recur_copy) =>  $1''
    destination_copy_path_dir=''$2''
    log -e ``Destination dir path: $destination_copy_path_dir''
    dep_list=`ldd $1 | perl -p -e 's/[^=]*=> ([^\s]*).*/$1/g' | egrep '^\/.*'`
    dep_arr=($dep_list)

    for dep_file_path in ``${dep_arr[@]}''
    do
        log -e ``\t\tdependency:  $dep_file_path''
        dep_file_name=`basename $dep_file_path`
        log $dep_file_name
        if [ ! -f $dep_file_name ]
        then
            log -e ``\t\t\tFile $dep_file_name does not exists''
            cp $dep_file_path $destination_copy_path_dir
            log -e ``\t\t\tcopied $dep_file_name TO $destination_copy_path_dir''

            recur_copy $dep_file_name $destination_copy_path_dir
        else
            log -e ``\t\t\tFile $FILE DOES exists''
            continue
        fi
    done    
    log -e ``#####\n'' 
}


## list_file_in_dir: takes two arguments lookup directory name and Destination path. Call recur_
copy function
list_file_in_dir()
{
    cd $1
    log ``present working directory `pwd`''
    echo ``lookup dir name $1''
    for file in $1/* ## iterate over all files within directory
    do
        filename=`basename $file`
        log -e ``*****\nFile Name(list_files_in_dir) =>  $file''
        recur_copy $filename $2
        log -e ``*****\n''
    done
}

## Input args processing
echo ``WARNING : This script does not work with relative path. Please specify full path when you 
pass arguments.''

if [[ $# -le 0 ]]; then
        echo ``Invalid arguments''
        echo ``$OPTIONS_HELP''
        break
fi
while test $# -gt 0; do
    case $1 in
       -v)
          debug=1
          #log ``some text''
        ;;
       -h)
            echo ``$OPTIONS_HELP''
            break
        ;;
        *)
        if [ $# -eq 2 ] && [ ! -z $1 ] && [ ! -z $2 ]
            then
                log ``two args found''
                LOOKUP_DIR=''$1''
                DEST_DIR=''$2''
                log ``Lookup_Dir = $LOOKUP_DIR''
                log ``Destination_Dir = $DEST_DIR''
                list_file_in_dir $1 $2
            else
                echo ``Invalid arguments''
                echo -e ``\nUsage: `basename $0` -h for help'';
                echo ``$OPTIONS_HELP''
            fi
            break
        ;;
    esac
    shift 
done



\end{lstlisting}

