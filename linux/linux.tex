\section{Linux and FreeBSD}

Before starting, make sure you have 


\subsection{Installing R}

If your distribution is Debian-derived, including Debian, Ubuntu, and Mint:
\begin{lstlisting}[title=Installing OpenMPI on Debian Linux]
apt-get install r-base-dev
\end{lstlisting}

If your distribution is ``Redhat-ish'', including Fedora and CentOS:
\begin{lstlisting}[title=Installing OpenMPI on Fedora Linux]
yum install R-devel
\end{lstlisting}

If your distribution is OpenSUSE:
\begin{lstlisting}[title=Installing OpenMPI on OpenSUSE Linux]
zypper install R-patched-devel
\end{lstlisting}

If you are using FreeBSD:
\begin{lstlisting}[title=Installing OpenMPI on FreeBSD]
cd /usr/ports/math/R && make install clean
\end{lstlisting}


\subsection{Installing MPI}

For these systems, we recommend using OpenMPI.  To install OpenMPI

If your distribution is Debian-derived, including Debian, Ubuntu, and Mint:
\begin{lstlisting}[title=Installing OpenMPI on Debian Linux]
apt-get install openmpi-bin libopenmpi-dev
\end{lstlisting}

If your distribution is ``Redhat-ish'', including Fedora and CentOS:
\begin{lstlisting}[title=Installing OpenMPI on Fedora Linux]
yum install openmpi openmpi-devel
\end{lstlisting}

If your distribution is OpenSUSE:
\begin{lstlisting}[title=Installing OpenMPI on OpenSUSE Linux]
zypper install openmpi-devel lam-devel
\end{lstlisting}

If you are using FreeBSD:
\begin{lstlisting}[title=Installing OpenMPI on FreeBSD]
cd /usr/ports/net/openmpi && make install clean
\end{lstlisting}




You can test the installation of OpenMPI via the command:

\begin{lstlisting}[title=Shell Command]
mpiexec - np 2 hostname
\end{lstlisting}



\subsection{Installing pbdR}

Installing pbdR should be fairly straight forward.


