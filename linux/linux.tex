\section{Linux and FreeBSD}

Before starting, you may need root access to your machine.  Also, you will need to know how to do some simple things via the terminal.  If you're using a standard Linux desktop, you probably have a terminal launcher in your applications menu somewhere.  If you're using some kind of weirdo tiling thing from 1990, then I assume you know what you're doing.  Additionally, if you are inexperienced with using the terminal, you should consider skimming \href{http://community.linuxmint.com/tutorial/view/100}{this short introduction}.

On Linux, unless you have a specific reason not to (in which case, most of this document is probably unnecessary for you), we recommend that you install \proglang{R} and MPI through your distribution's package repository (especially MPI).  This will make the installation process \emph{much} simpler, and generally ``just works''.

If instructions for your favorite distribution are not listed below, we would be happy to incorporate submissions/corrections.

Finally, if you are completely new to \proglang{R}, then you might consider reading the \href{http://cran.r-project.org/doc/FAQ/R-FAQ.html}{R FAQ}.  To learn more about programming with \proglang{R}, then you may find the \href{http://cran.us.r-project.org/doc/manuals/R-intro.html}{Introduction to R} guide useful.






\subsection{Installing R}

You can install \proglang{R} either from your package repo (recommended) or from source.

\subsubsection{Installing from a Package Repository}

If your distribution is Debian-derived, including Debian, Ubuntu, and Mint:
\begin{lstlisting}[language=sh]
apt-get install r-base-dev
\end{lstlisting}

\vspace{.4cm}
If your distribution is ``Redhat-ish'', including Redhat, Fedora, and CentOS:
\begin{lstlisting}[language=sh]
yum install R-devel
\end{lstlisting}

\vspace{.4cm}
If your distribution is OpenSUSE:
\begin{lstlisting}[language=sh]
zypper install R-patched-devel
\end{lstlisting}

\vspace{.4cm}
If you are using FreeBSD:
\begin{lstlisting}[language=sh]
cd /usr/ports/math/R && make install clean
\end{lstlisting}



\subsubsection{Compiling from Source}

You can find \proglang{R} sources from \url{http://cran.r-project.org/sources.html}

Start by opening a terminal and navigate to the folder containing the \proglang{R} source package you just downloaded.  You can extract the archive by executing, for example
\begin{lstlisting}[language=sh]
tar zxvf R-3.0.0.tar.gz
\end{lstlisting}

From here, generally it should be enough to simply execute
\begin{lstlisting}[language=sh]
./configure && make && make install
\end{lstlisting}
without problems.









\subsection{Installing MPI}

You can install \proglang{R} either from your package repo (recommended) or from source.

\subsubsection{Installing from a Package Repository}

For these systems, we recommend using OpenMPI.  To install OpenMPI

If your distribution is Debian-derived, including Debian, Ubuntu, and Mint:
\begin{lstlisting}[language=sh]
apt-get install openmpi-bin libopenmpi-dev
\end{lstlisting}

\vspace{.4cm}
If your distribution is ``Redhat-ish'', including Fedora and CentOS:
\begin{lstlisting}[language=sh]
yum install openmpi openmpi-devel
\end{lstlisting}

\vspace{.4cm}
If your distribution is OpenSUSE:
\begin{lstlisting}[language=sh]
zypper install openmpi-devel lam-devel
\end{lstlisting}

\vspace{.4cm}
If you are using FreeBSD:
\begin{lstlisting}[language=sh]
cd /usr/ports/net/openmpi && make install clean
\end{lstlisting}



\subsubsection{Compiling from Source}

If you want to install OpenMPI from source (I don't really recommend this unless you think you have a good reason to), then the sources are available \href{http://www.open-mpi.org/software/ompi/v1.6/}{here}.









\subsection{Installing pbdR Packages}
Installing pbdR should go smoothly.  The simplest way to install the packages is from an \proglang{R} terminal, which will manage dependencies for you much like your distro's package manager.  Additionally, our packages are available in the Fedora repositories.


\subsubsection{Installing from CRAN}
This is perhaps the simplest way to proceed, as \proglang{R} will handle any package dependency resolution for you.  Simply start an \proglang{R} session (from the terminal, type \code{R} then press enter) and issue the command:
\begin{lstlisting}[language=rr]
install.packages(<package>)
\end{lstlisting}
So for example, to install \pkg{pbdMPI}, you might execute:
\begin{lstlisting}[language=rr]
install.packages(pbdMPI)
\end{lstlisting}


\subsubsection{Installing from the Shell}
If you have downloaded a pbdR (or other \proglang{R}) package, then installing from the shell simply amounts to issuing the command:
\begin{lstlisting}[language=sh]
R CMD INSTALL <package>
\end{lstlisting}
So for example, to install \pkg{pbdMPI}, you might execute:
\begin{lstlisting}[language=sh]
R CMD INSTALL pbdMPI_0.1-6.tar.gz
\end{lstlisting}


\subsubsection{Installing from Github}
CRAN policy is such that updates to packages can not be made too frequently.  For this reason, the development versions of our packages will have bugfixes and new features much more quickly than CRAN versions.  

The easiest way to install from github is using Hadley Wichkam's \pkg{devtools} package (which can be installed via \code{install.packages(devtools)}).  Assuming you have this package installed, then from an \proglang{R} session, to install a pbdR package you would issue one of the following:

\begin{lstlisting}[language=rr]
library(devtools)

install_github(repo="pbdMPI", username="RBigData")
install_github(repo="pbdSLAP", username="RBigData")
install_github(repo="pbdNCDF4", username="RBigData")
install_github(repo="pbdNCDF4", username="RBigData")
install_github(repo="pbdBASE", username="RBigData")
install_github(repo="pbdDMAT", username="RBigData")
install_github(repo="pbdDEMO", username="RBigData")
\end{lstlisting}

You can also install \emph{really} new package builds, which will be very current in terms of features, but also bugs (or even complete package breakage).  If you're sure you want these packages, then you can install them as follows:

\begin{lstlisting}[language=rr]
# dev repo 1
install_github(repo="pbdMPI", username="snoweye")
install_github(repo="pbdSLAP", username="snoweye")
install_github(repo="pbdNCDF4", username="snoweye")
# dev repo 2
install_github(repo="SEXPtools", username="wrathematics")
install_github(repo="pbdBASE", username="wrathematics")
install_github(repo="pbdDMAT", username="wrathematics")
install_github(repo="pbdDEMO", username="wrathematics")
\end{lstlisting}


\section{R and pbdR installation on Cray system }\label{sec:Rinstall}

Below is installation instructions of \proglang{R} on Cray machine. 
Prerequisites for R and pbdR software installation on Cray machine: GCC, Cray-MPICH and ACML/MKL.

We use following software tools/packages: Programming Environment GNU-4.2.34 with \pkg{GCC}-4.8.2, \pkg{Cray-MPICH}-6.2.0, \pkg{ACML}-5.3.1, and \proglang{R} 3.0.2(compied dynamically). The \proglang{R} packages and their corresponding versions used in these: \pkg{rlecuyer} 0.3-3, \pkg{memuse} 0.2-0, \pkg{pbdMPI} 0.2-2, \pkg{pbdSLAP} 0.1-8, \pkg{SEXPtools} 0.1-0, \pkg{pbdBASE} 0.3-0, and \pkg{pbdDMAT} 0.3-0.

\vspace*{-.5cm}
\begin{lstlisting}[language=bashy]

## Load/set appropriate modules and software environments 
module swap PrgEnv-cray PrgEnv-gnu/4.2.34
module load java/jdk1.7.0_45
module load acml/5.3.1

## Create installation directory on lustre space
mkdir -p /scratch/sciteam/\$USER/R-3.0.2-Install
export R_WORK_HOME=/scratch/sciteam/\$USER/R-3.0.2-Install
cd \$R_WORK_HOME

## Download R source code 
wget http://mirrors.nics.utk.edu/cran/src/base/R-3/R-3.0.2.tar.gz
tar -xzvf R-3.0.2.tar.gz
cd R-3.0.2

## Configure R
./configure --prefix=\$R_WORK_HOME --enable-R-profiling --enable-memory-profiling --enable-R-shlib --enable-BLAS-shlib --enable-lto --enable-byte-compiled-packages --enable-shared --enable-long-double --with-readline --with-tcltk --with-cairo --with-libpng --with-jpeglib --with-libtiff --with-system-pcre --with-valgrind-instrumentation --with-x --with-blas=``-I/opt/acml/5.3.1/gfortran64_mp/include -L/opt/acml/5.3.1/gfortran64_mp/lib -lacml_mp'' --with-lapack > configure.log

make > make.log
make check > make_check.log
make install > make_install.log
make check-all > make_check_all.log

## Install R packages
export PATH=\$R_WORK_HOME/bin:\$PATH
export LD_LIBRARY_PATH=\$R_WORK_HOME/lib64/R/lib:\$LD_LIBRARY_PATH

mkdir -p \$R_WORK_HOME/R_pkg_sources
cd \$R_WORK_HOME/R_pkg_sources


## Download R packages
wget http://cran.r-project.org/src/contrib/rlecuyer_0.3-3.tar.gz
wget http://cran.r-project.org/src/contrib/pbdMPI_0.2-2.tar.gz
wget http://cran.r-project.org/src/contrib/pbdSLAP_0.1-8.tar.gz
wget https://github.com/wrathematics/pbdBASE/archive/master.zip
mv master pbdBASE.zip
unzip pbdBASE.zip

wget https://github.com/wrathematics/pbdDMAT/archive/master.zip
mv master pbdDMAT.zip
unzip pbdDMAT.zip

wget  https://github.com/wrathematics/SEXPtools/archive/master.zip
mv master SEXPtools.zip
unzip SEXPtools.zip

wget https://github.com/wrathematics/memuse/archive/master.zip
mv master memuse.zip
unzip memuse.zip

## R packages installation
R CMD INSTALL --no-test-load rlecuyer_0.3-3.tar.gz

## pbdMPI install
R CMD INSTALL --no-test-load pbdMPI --configure-args=``--with-mpi=/opt/cray/mpt/6.2.0/gni/mpich2-gnu/48/ --with-mpi-type=MPICH3''

R CMD INSTALL --no-test-load pbdSLAP_0.1-8.tar.gz
 
R CMD INSTALL --no-test-load pbdBASE

R CMD INSTALL --no-test-load pbdDMAT

R CMD INSTALL --no-test-load SEXPtools

R CMD INSTALL --no-test-load memuse

## Copy all needed dynamic libraries to Lustre 

mkdir -p \$R_WORK_HOME/system_libs
cd \$R_WORK_HOME/
./dyn_libs_copy.sh \$R_WORK_HOME/lib64/R/lib \$R_WORK_HOME/system_libs
./dyn_libs_copy.sh \$R_WORK_HOME/lib64/R/library/pbdMPI/libs \$R_WORK_HOME/system_libs

echo ``Now you are ready to use \proglang{R} and \proglang{pbdR}. Good Luck !!!''

#*********************
#*********************
## Below is source code of `'dyn_libs_copy.sh''
## This script uses ldd and copy all needed shared object files recursively
OPTIONS_HELP='
Command line options:
     -h Print this help menu
     -v Verbose (lookup_dir_Path destination_dir_path)

Usage: ./dyn_libs_copy.sh lookup_dir_Path destination_dir_path
Usage Example:
./dyn_libs_copy.sh /package/version/os_compiler/bin /package/version/os_compiler/system_libs
./dyn_libs_copy.sh -v /package/version/os_compiler/lib /package/version/os_compiler/system_libs

Verbose mode: ./dyn_libs_copy.sh -v lookup_dir_Path destination_dir_path
'

# Debug function
log(){
if [[ \$debug -eq 1 ]]; then
        echo ``\$@''
    fi
}

## recur_copy function: takes two arguments: filename and Destination path
recur_copy()
{
    log -e ``#####\nFile Name(recur_copy) =>  $1''
    destination_copy_path_dir=''$2''
    log -e ``Destination dir path: $destination_copy_path_dir''
    dep_list=`ldd $1 | perl -p -e 's/[^=]*=> ([^\s]*).*/$1/g' | egrep '^\/.*'`
    dep_arr=($dep_list)

    for dep_file_path in ``${dep_arr[@]}''
    do
        log -e ``\t\tdependency:  $dep_file_path''
        dep_file_name=`basename $dep_file_path`
        log $dep_file_name
        if [ ! -f $dep_file_name ]
        then
            log -e ``\t\t\tFile $dep_file_name does not exists''
            cp $dep_file_path $destination_copy_path_dir
            log -e ``\t\t\tcopied $dep_file_name TO $destination_copy_path_dir''

            recur_copy $dep_file_name $destination_copy_path_dir
        else
            log -e ``\t\t\tFile $FILE DOES exists''
            continue
        fi
    done    
    log -e ``#####\n'' 
}


## list_file_in_dir: takes two arguments lookup directory name and Destination path. Call recur_
copy function
list_file_in_dir()
{
    cd $1
    log ``present working directory `pwd`''
    echo ``lookup dir name $1''
    for file in $1/* ## iterate over all files within directory
    do
        filename=`basename $file`
        log -e ``*****\nFile Name(list_files_in_dir) =>  $file''
        recur_copy $filename $2
        log -e ``*****\n''
    done
}

## Input args processing
echo ``WARNING : This script does not work with relative path. Please specify full path when you 
pass arguments.''

if [[ $# -le 0 ]]; then
        echo ``Invalid arguments''
        echo ``$OPTIONS_HELP''
        break
fi
while test $# -gt 0; do
    case $1 in
       -v)
          debug=1
          #log ``some text''
        ;;
       -h)
            echo ``$OPTIONS_HELP''
            break
        ;;
        *)
        if [ $# -eq 2 ] && [ ! -z $1 ] && [ ! -z $2 ]
            then
                log ``two args found''
                LOOKUP_DIR=''$1''
                DEST_DIR=''$2''
                log ``Lookup_Dir = $LOOKUP_DIR''
                log ``Destination_Dir = $DEST_DIR''
                list_file_in_dir $1 $2
            else
                echo ``Invalid arguments''
                echo -e ``\nUsage: `basename $0` -h for help'';
                echo ``$OPTIONS_HELP''
            fi
            break
        ;;
    esac
    shift 
done



\end{lstlisting}


