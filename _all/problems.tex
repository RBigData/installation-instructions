\section{Installation Problems}

During the course of installation, you may run into unrecoverable issues.  The pbdR team does not support MPI libraries or R core, so if you have problems during that portion of the installation phase, we probably can not directly help you.  However, there are still many great resources at your disposal, maintained by those individual projects.

\subsection{R and MPI}

If you have problems installing or customizing R, see the \emph{R Installation and Administration Manual} at \url{http://cran.r-project.org/doc/manuals/R-admin.html} for help.

If you are having trouble installing an MPI library, you should see that library's official documentation.  For OpenMPI, see \url{http://www.open-mpi.org/community/help/} and for MPICH, see \url{http://www.mpich.org/documentation/guides/} .

For the remainder, we will be addressing installation issues with pbdR packages.

\subsection{pbdR}\label{sec:porblam}

This is a quick list of potential problems you could encounter when installing pbdR packages.  For additional troubleshooting or installation options, each package has a vignette which may offer additional useful information.

\begin{itemize}
  \item \textbf{When compiling pbdMPI from source}, you may be required to pass a configure argument at compile time.  So for example, if you have OpenMPI installed and were installing from the command line, then you would issue the command:
\begin{Code}
R CMD INSTALL pbdMPI_0.1-6.tar.gz \
        --configure-args='--with-mpi-type=OPENMPI'
\end{Code}
  or if installing from R:
  \begin{lstlisting}[language=rr]
install.packages("pbdMPI", configure.args='--with-mpi-type=OPENMPI')
\end{lstlisting}
  See the \pkg{pbdMPI} vignette for more details.

  
  \item \textbf{If you are installing on a cluster} where you must install on the login node which can not execute \code{mpirun}, then pass the install option \code{--no-test-load}.    So for example, if installing from the command line, then you would issue the command:
\begin{Code}
R CMD INSTALL pbdMPI_0.1-6.tar.gz --no-test-load
\end{Code}
  or if installing from R:
  \begin{lstlisting}[language=rr]
install.packages("pbdMPI", INSTALL_opts='--no-test-load')
\end{lstlisting}

  
  \item \textbf{If you are installing binaries on MAC OS X}, do not use the gui.  You can install from source using the gui, or you can install binaries (or from source) using the terminal.  But you can not install binaries using the gui.  So if you want to install binaries,  you should open Finder, then navigate to \code{Applications/Utilities/} and select \code{Terminal}.  Next, type \code{R} and press enter.  Now try to install the packages.

\end{itemize}
